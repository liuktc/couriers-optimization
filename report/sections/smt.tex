\section{SMT model}

The SMT model is based on the CP model, and follows the same definition of paths of the couriers. 

\subsection{Decision variables}

The SMT model relies on the following decision variables, where boolean variables are from the Propositional Logic Theory, and integer variables are from LIA Theory.

\begin{itemize}
    \item For each package $j$, $A_j \in \{1, \dots, m\}$ (\texttt{ASSIGNMENTS[j]} in Z3) indicates which courier delivers package $j$. More specifically, $A_j = i$ iff the courier $i$ delivers package $j$.
    
    \item For each courier $i$, $P_{i,d} \in \{1, \dots, n + 1\}$ for $d \in \{1, \dots, n + 1\}$ (\texttt{PATH[i][d]} in Z3) models the path taken by the courier $i$. The path is defined such that $P_{i, d_1} = d_1$ iff the location $d_1$ is not part of the route of the courier $i$ and $P_{i, d_1} = d_2$ iff $d_2$ is visited immediately after $d_1$ in the route of $c$. In other words, each relevant location indicates which is its successor and the overall route is defined as a Hamiltonian cycle that starts from $n+1$ (i.e., the depot).

\end{itemize}

\subsection{Auxiliary variables}
More variables have been defined in order to make it easier to define some constraints and the objective function:

\begin{itemize}
    \item For each courier $i$, $D_i \in \mathbb{N}$ (\texttt{DISTANCES[i][d]} in Z3) is equal to the distance traveled by each courier.

    \item For each courier $i$, $C_i \in \mathbb{N}$ (\texttt{COUNT[i]} in Z3) models the number of packages delivered by courier $i$.

    \item For each item $j$ and for each courier $i$, $PPC_{i,j}$ (\texttt{PACKS\_PER\_COURIER[i][j]} in Z3) model the packages delivered by each courier and it is used for symmetry breaking only.
\end{itemize}

\subsection{Objective function}
The objective function is defined as follows:

\[ \texttt{obj} = \max_{i \in \{1, \dots, m\}} D_i  \]

\subsection{Constraints}

\begin{itemize}
    \item All the elements of each row of $P$ should be distinct:

    \begin{equation}
        \forall i \in \{1, \dots, m\}: \quad P_{i,j_1} \neq P_{i,j_2} \quad \forall j_1,j_2 \in \{1, \dots, n+1\}, j_1 \neq j_2 
    \end{equation}

    \item Count constraint: 

    \begin{equation}
        \forall i \in \{ 1, \dots, m \}: \quad C_i = | \{ j \in \{1, \dots, n\} | A_j = i \}|
    \end{equation}


    \item Weight constraint: 
    \begin{equation}
        \forall i \in \{ 1, \dots, m \}: \sum_{j \in \{1, \dots, n\}: A_j = i} s_j \leq l_i 
    \end{equation}

    \item Subcircuit constraint: each row $P_i$ should define a subcircuit, that is a Hamiltonian path that ignores all the elements $P_{i,j} = j$, namely the packages that the courier $i$ don't deliver.

    \item Distance constraint:

    \begin{equation}
        \forall i \in \{1, \dots, m\}: \quad D_i = \sum_{j \in \{1, \dots, n+1\}} \begin{cases*}
                    D_{j,P_{i,j}} & if  $P_{i,j} \neq j$  \\
                    0 & if $P_{i,j} = j$
                 \end{cases*}
    \end{equation}

    \item Assignment and path constraint:
    \begin{equation}
        \forall j \in \{1, \dots, n\}: \quad A_j = i \Longleftrightarrow P_{i,j} \neq j
    \end{equation}

    \begin{equation}
        \forall j \in \{1, \dots, n\}: \quad A_j \neq i \Longleftrightarrow P_{i,j} = j
    \end{equation}
\end{itemize}

\subsubsection{Symmetry breaking constraints}
See Symmetry breaking constraint of CP, section 2.3.1. 

\subsubsection{Implied constraints}

\begin{itemize}
    \item In order to ensure that each courier delivers at least one package:
    \begin{equation}
        \forall i \in \{1, \dots, m\}: \quad C_i \geq 1  
    \end{equation}
\end{itemize}


\subsection{Validation}

\subsubsection{Experimental design}
The model was implemented in Z3 using the Python API, \texttt{z3py}. All experiments were conducted on : ????. Due to memory limitations, problem instances after the 10th required more than 5GB of RAM. For this reason, the memory available waas increased to 8GB to ensure the successful computation of the results.

The basic model, using a single solver, failed to find a solution in a reasonable amount of time for problem instances after the 10th. To address this limitation and find a solution, even if suboptimal, for challenging instances, two strategies have been implemented:

\begin{itemize}
    \item \textbf{Two Solvers Approach}
    \begin{itemize}
        \item \textbf{Description}: The first solver  finds $A$ and the second one finds $P$ given $A$. In other words, the first solver decides which courier delivers which package and the second solver decides the route taken by each courier, basically solving $m$ different Travelling Salesman Problems. 
        \item \textbf{Performance}: This model outperform the single solver model, but still struggle to find a solution in a reasonable amount of time for the bigger instances.
    \end{itemize}

    \item \textbf{Local Search Approach}:
    \begin{itemize}
        \item \textbf{Description}: This approach also uses two solver. The first one finds $A$ and the second one finds $P$ given $A$. However, instead of finding an optimal solution for $P$, the second solver performs a local search starting from a trivial solution.
        \item \textbf{Performance}: The local search strategy enables the model to find a solution in a reasonable amount of time, even for the largest instances.
    \end{itemize} 

\end{itemize}

In addition to using \texttt{z3py}, some experiments with SMT-LIB were conducted, in order to expolore different solvers easily. To support these experiments, a Python library was developed that automates the generation of SMT-LIB code from high level python interface. This allowed us to create and test different model with different solvers in a very fast and easy way. Nevertheless, the results are pretty similar to the one obtained using the Z3 solver through the \texttt{z3py} library.

\subsubsection{Experimental results}


\begin{table}[h]
	\centering
	\caption{Caption}
	\begin{tabular}{cccccccc}
		\toprule
		Id & plain & plain-symm & plain-impl & twosolver & ts-symm & ts-impl & local-search \\ 
		\midrule
		1 & \textbf{14} & 	\textbf{14} & 	\textbf{14} & 	\textbf{14} & 	\textbf{14} & 	\textbf{14} & 	\textbf{14} \\ 
		2 & \textbf{226} & 	\textbf{226} & 	\textbf{226} & 	226 & 	226 & 	226 & 	226 \\ 
		3 & \textbf{12} & 	\textbf{12} & 	\textbf{12} & 	\textbf{12} & 	\textbf{12} & 	\textbf{12} & 	\textbf{12} \\ 
		4 & \textbf{220} & 	\textbf{220} & 	\textbf{220} & 	220 & 	-- & 	220 & 	220 \\ 
		5 & \textbf{206} & 	\textbf{206} & 	\textbf{206} & 	\textbf{206} & 	\textbf{206} & 	\textbf{206} & 	\textbf{206} \\ 
		6 & \textbf{322} & 	\textbf{322} & 	\textbf{322} & 	322 & 	-- & 	322 & 	322 \\ 
		7 & 174 & 	168 & 	181 & 	167 & 	-- & 	167 & 	167 \\ 
		8 & \textbf{186} & 	\textbf{186} & 	\textbf{186} & 	186 & 	186 & 	186 & 	186 \\ 
		9 & \textbf{436} & 	\textbf{436} & 	\textbf{436} & 	436 & 	436 & 	436 & 	436 \\ 
		10 & \textbf{244} & 	\textbf{244} & 	\textbf{244} & 	244 & 	244 & 	244 & 	244 \\ 
		11 & -- & 	-- & 	-- & 	-- & 	-- & 	-- & 	547 \\ 
		12 & -- & 	-- & 	-- & 	-- & 	-- & 	-- & 	435 \\ 
		13 & -- & 	-- & 	-- & 	1812 & 	-- & 	1832 & 	632 \\ 
		14 & -- & 	-- & 	-- & 	-- & 	-- & 	-- & 	1177 \\ 
		15 & -- & 	-- & 	-- & 	-- & 	-- & 	-- & 	1140 \\ 
		16 & -- & 	-- & 	-- & 	1510 & 	-- & 	1861 & 	303 \\ 
		17 & -- & 	-- & 	-- & 	-- & 	-- & 	-- & 	1525 \\ 
		18 & -- & 	-- & 	-- & 	-- & 	-- & 	-- & 	917 \\ 
		19 & -- & 	-- & 	-- & 	-- & 	-- & 	-- & 	398 \\ 
		20 & -- & 	-- & 	-- & 	-- & 	-- & 	-- & 	1378 \\ 
		21 & -- & 	-- & 	-- & 	-- & 	-- & 	-- & 	648 \\ 
		\bottomrule
	\end{tabular}
\end{table}






    
