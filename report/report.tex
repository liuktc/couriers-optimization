\documentclass{article}
\usepackage{array}
\usepackage{graphicx}
\usepackage{amsmath, amsthm, thmtools, mathdots, mathtools, amssymb}
\usepackage[pdfusetitle]{hyperref}
\usepackage[all]{hypcap} % Links hyperref to object top and not caption
\usepackage[nameinlink]{cleveref}   
\usepackage{makecell, multirow, booktabs}
\usepackage{caption, subcaption}
\usepackage{float}
\usepackage[bottom]{footmisc}
\usepackage[inline]{enumitem}
\usepackage{appendix}
\usepackage{biblatex}
\addbibresource{references.bib}
\hypersetup{ colorlinks, citecolor=black, filecolor=black, linkcolor=black, urlcolor=black, linktoc=all }
\newcommand*\rot{\rotatebox{90}}

\let\endtitlepage\relax

\newtheorem{theorem}{Theorem}
\newtheorem{definition}{Definition}
\newtheorem{lem}{Claim}


\begin{document}
    \begin{titlepage}
        \begin{center}
            {\LARGE Multiple Couriers Problem}
            \vspace*{1em}
            
            Valerio Costa, Luca Domeniconi, Claudia Maiolino, Tian Cheng Xia

            \centerline{\{valerio.costa, luca.domeniconi5, claudia.maiolino, tiancheng.xia\}@studio.unibo.it}
        \end{center}
    \end{titlepage}

    \thispagestyle{plain}

    \section{Introduction} \label{sec:intro}
    The problem of this project is known in the literature as the capacitated vehicle routing problem and can be easily proven to be NP-hard. We tackle this problem by virtually splitting it into two sub-problems: first we look for an assignment of the items to the couriers and then search for the routes of each of them (i.e., by solving multiple travelling salesman problems). To solve the latter, we follow the approach presented in \cite{vrp} where the route of a courier is modelled through the variables $P_d \in [1, n+1]$ with $d \in [1, n+1]$ defined as follows:
    \begin{equation}
        \label{eq:path_def}
        P_{d_1} = \begin{cases}
            d_2 & \text{iff the location $d_2 \neq d_1$ is visited immediately after $d_1$}\\
            d_1 & \text{iff $d_1$ is not part of the route of the courier}
        \end{cases}
    \end{equation} 
    With proper assignment and subtour elimination constraints, $P_d$ allows to define a Hamiltonian cycle that passes through the items that the courier delivers and the solution can be extracted by traversing the cycle starting from the depot $n+1$.

    The lower-bound of the objective function is common to all models and is defined as the maximum path cost that involves a single package:
    \begin{equation}
        \max_{p \in [1, n]} \left\{ D_{n+1, p} + D_{p, n+1} \right\}
    \end{equation}

    As upper-bound, also common to all models, we observed that it does not provide any improvement to the results. Nevertheless, we defined it as:
    \begin{equation}
        \sum_{d_1 \in [1, n+1]} \max_{d_2 \in [1, n+1]} D_{d_1, d_2}
    \end{equation}

    Symmetry breaking constraints are also common to all models. By considering couriers with the same capacity, the following constraints can be used to avoid symmetries:
    \begin{itemize}
        \item By imposing an ordering on the amount of assigned packages:
            \begin{equation}
                \label{eq:cp_symm_amount}
                \forall c_1, c_2 \in [1, m]: (c_1 < c_2 \land l_{c_1} = l_{c_2}) \Rightarrow Q_{c_1} \leq Q_{c_2}
            \end{equation}
            where $Q_c$ is the amount of packages delivered by the courier $c$.
        \item By imposing an ordering on the indexes of the assigned packages:
            \begin{equation}
                \label{eq:cp_symm_packs}
                \forall c_1, c_2 \in [1, m]: (c_1 < c_2 \land l_{c_1} = l_{c_2}) \Rightarrow A_{c_1} <_\texttt{lex} A_{c_2}
            \end{equation}
            where $A_c$ is an ordered vector containing the packages delivered by the courier $c$.
    \end{itemize}

    As the triangle inequality holds, we also identified an implied constraint that consists of imposing that each courier delivers at least a package (a short proof is provided in \Cref{sec:impl_proof}):
    \begin{equation}
        \label{eq:impl_constr}
        \forall c \in [1, m]: Q_c \geq 1  
    \end{equation}
    where $Q_c$ is the amount of packages delivered by the courier $c$.
    This obviously is applicable only if each courier is able to carry at least an item.

    All experiments were done using the same random seed and were run as workflows on GitHub Actions which provides two cores at 2.45 GHz and 7 GB of memory. To guarantee a safe margin for the Docker container to run, the actual usable memory was capped to 5 GB.

    The work has been completed in approximately one month and has been roughly split in the following way: Xia did the CP part, Costa worked on SAT, Domeniconi did the SMT models, and Maiolino completed the MIP part. The main difficulties we encountered are the following: 
    \begin{enumerate*}[label=(\roman*)]
        \item lack of proper documentation for many tools we used,
        \item difficulties to find a more efficient way to solve bigger instances,
        \item the need of time to run the experiments on all instances.
    \end{enumerate*}


    \section{CP model}


\subsection{Decision variables}


\subsection{Objective function}


\subsection{Constraints}


\subsection{Validation}
    \section{SAT model}

% The resolution of the problem is approached following two models.

% \begin{itemize}
%     \item \textbf{Unified model}:
%     based on the definition of two decision variables \texttt{assignments} and \texttt{paths}.
%     \item \textbf{Matrix model}:
%     based on the definition of one decision variable $X$.
% \end{itemize}

% Some modifications were applied to those model in order to visualize eventual differences in terms of performances.

% The discussion is going to be much more focused on the Unified model just to mantain a logical thread within the explication of models used for other solvers.

% \subsection{Unified Model}

% \paragraph*{General Definition}
% The Unified model aims to find the correct assignment to both decision variables, in such a way that it satisfy all constraints.\\
% The idea behind this kind of model is based on the separation of the two task just specified in one.

% \paragraph*{Original Workflow}
% The original workflow follows the next steps:
% \begin{enumerate}
%     \item find satisfying values for the \texttt{assignments} variable, else end the algorithm
%     \item find satisfying values for the \texttt{paths} variable, else proceed to step $4$
%     \item repeat step $2$ to find a new optimized solution,
%     \item repeat step $1$ just to find another assignment
% \end{enumerate}

% \paragraph*{New Workflow}
% The new algorithm simply tries to optimize the assignment to both variable.

% \paragraph*{Pro}
% The model was designed to improve certain intrinsic problems of the definition of a problem through SAT:
% \begin{itemize}
%     \item \textbf{Specificity of constraints}: constraining much more the assignments of the two variable guarantes to mantain lower width of exploration of the resolution tree.
% \end{itemize} 

% \paragraph*{Contro}
% The model falls into a few issues such:

% \begin{itemize}
%     \item \textbf{Dimension}: being based on two decision variables, the dimension of the problem scale exponentially with them. \footnote{The scaling dimension of the problem implies higher needs of time to build the model.}
% \end{itemize}

\subsection{Decision variables}

For SAT, we defined two different models:
\begin{description}
    \item[Unified model] based on the definition of the two decision variables $A$ (\texttt{assignments} in Z3) and $P$ (\texttt{paths} in Z3) as defined in \Cref{sec:intro}.

    \item[Matrix model] based on the definition of a single matrix $X$ such that $X\texttt{[$c$, $p$, $k$]} = 1$ iff courier $c$ delivers item $p$ as its $k$-th package.
\end{description}

Our discussion will be focused on the former to maintain coherence with the models defined in the other methods and obtain more comparable results. It decision variables are the following:

\begin{itemize}
    \item $A$ is an $n \times m$ matrix such that $A\texttt{[$p$, $c$]} = 1$ iff courier $c$ delivers item $p$.

    \item $P$ is an $m \times (n+1) \times (n+1)$ matrix such that $P\texttt{[$c$, $loc_1$, $loc_2$]} = 1$ iff courier $c$ moves from location $loc_1$ to $loc_2$.

    \item $U$ is an $m \times n \times n$ matrix to implement the MTZ subtour elimination \cite{mtz_subtour}. $U\texttt{[$c$, $p$, $k$]} = 1$ iff for the courier $c$ the item $p$ is delivered as the $k$-th.
\end{itemize}

\subsubsection{Objective function}

The objective function is computed as follows:
\begin{equation}
    \label{eq:obj_fun}
    \max_{c \in \{ 1, \dots, m \}}
    \sum_{loc_1=1}^{n+1} \sum_{loc_2=1, loc_2 \neq loc_1}^{n+1} \texttt{D[$loc_1$, $loc_2$]} \cdot P\texttt{[$c$, $loc_1$, $loc_2$]}
\end{equation}


\subsubsection{Constraints}

\paragraph*{Assignment related constraints}

\begin{itemize}
    \item Capacity constraint:
    \begin{itemize}
        \item Sum of sizes of packs delivered by a singular courier must be under its load limit
        \begin{equation}
            \label{eq:capacity1}
            \forall c \in [1, m]:
            \sum_{p=1}^{n} A\texttt{[$p$, $c$]} \cdot s\texttt{[$p$]} \leq \texttt{$l$[$c$]}
        \end{equation}
        \item Each pack must be delivered only by a courier
        \begin{equation}
            \label{eq:capacity2}
            \forall p \in [1, n], \exists! c \in [1, m]: A\texttt{[$p$, $c$]} = 1
        \end{equation}
    \end{itemize}
\end{itemize}

\paragraph*{Path related constraints}

\begin{itemize}
    \item General path constraints
    \begin{itemize}
        \item If courier delivers at least one package, there must exist a destination from DEPOT\footnote{DEPOT is the original position, hypotesized as \texttt{n+1}.} with true value, else it can stay in DEPOT
        \begin{equation}
            \label{eq:gen_path_constr1}
            \forall c \in [1, m],
            \forall p \in [1, n]:
            \begin{cases}
                P\texttt{[$c$, DEPOT, DEPOT]}=0 & \texttt{if } \sum_{p=1}^{n} A\texttt{[$p$, $c$]} \geq \texttt{1}\\
                P\texttt{[$c$, DEPOT, DEPOT]}=1 & \texttt{if } \sum_{p=1}^{n} A\texttt{[$p$, $c$]} = \texttt{0} % or otherwise
            \end{cases}
        \end{equation}

        \item If courier delivers pack p, its destination must be different from p 
        \begin{equation}
            \label{eq:gen_path_constr2}
            \forall c \in [1, m],
            \forall p \in [1, n]:
            \begin{cases}
                P\texttt{[$c$, $p$, $p$]}=0 & \texttt{if } A\texttt{[$p$, $c$]}=1\\
                P\texttt{[$c$, $p$, $p$]}=1 & \texttt{if } A\texttt{[$p$, $c$]}=0 % or otherwise
            \end{cases}
        \end{equation}

        \item Each pack must be delivered by a single courier only once
        \begin{equation}
            \label{eq:gen_path_constr3}
            \forall c \in [1, m],
            \forall loc_1 \in [1, n+1]:
            \quad
            \sum_{loc_2=1}^{n+1} P\texttt{[$c$, $loc_1$, $loc_2$]} = 1
        \end{equation}
    \end{itemize}
    \item Subcircuit constraints
    \begin{itemize}
        \item Only a single courier must deliver pack in location $loc_2$ from $loc_1$
        \begin{equation}
            \label{eq:subtour_constr1}
            \forall c \in [1, m],
            \forall loc_2 \in [1, n+1]:
            \quad
            \sum_{loc_1=1}^{n+1} P\texttt{[$c$, $loc_1$, $loc_2$]} = 1
        \end{equation}
        \item Subtour elimination
        \begin{itemize}
            \item $u$ relative to first pack $p$ of courier $c$ path must have value = 1
            \begin{equation}
                \label{eq:subtour_constr2}
                \forall c \in [1, m],
                \forall p \in [1, n]:
                P\texttt{[$c$, DEPOT, $p$]}=1
                \Rightarrow
                U\texttt{[$c$, $p$, $1$]}=1
            \end{equation}
            \item $u_j \geq u_i + 1$
            \begin{equation}
                \label{eq:subtour_constr3}
                \forall c \in [1, m], \ \forall i, j, k \in [1, n]: \quad
                P\texttt{[$c$, $i$, $j$]} \land U\texttt{[$c$, $i$, $k$]} \Rightarrow
                \sum_{l=k+1}^{n} U\texttt{[$c$, $j$, $l$]} = 1
            \end{equation}
            
            \item Exatcly one true value for each U\texttt{[$c, p, :$]} vector
            \begin{equation}
                \label{eq:subtour_constr4}
                \forall c \in [1, m],
                \forall p_1 \in [1, n]:
                \sum_{p_2=1}^{n} U\texttt{[$c$, $p_1$, $p_2$]} = 1
            \end{equation}

            \item MTZ formulation constraint:
            \begin{center}
                $u_i - u_j + 1 \leq (n - 1)$ * $(1 - P[c, i, j] )$
            \end{center}
            \begin{equation}
                \label{eq:subtour_constr5}
                \forall c \in [1, m], \ \forall i, j, k_1, k_2 \in [1, n]: 
            \end{equation}
            \[
                U\texttt{[$c$, $i$, $k_1$]} \land U\texttt{[$c$, $j$, $k_2$]} \Rightarrow
                (k_1 - k_2 + 1) \leq (n-1) \cdot (1 - P\texttt{[$c$, $i$, $j$]})
            \]
        \end{itemize}
    \end{itemize}
\end{itemize}

\paragraph*{Symmetry breaking constraint}

A possible way to reduce tree exploration is to introduce symmetry breaking constraints.
One of them in this case can consist in constraining the order of assignments of packs between couriers with the same amount of load capacity.

But the experimentation didn't lead to major improvements.

\subsection{Validation}

\subsubsection*{Experimental design}

Some modifications were applied to the basic model in order to visualize eventuale differences in performances.
The original Unified Model was modified in three different versions:
\begin{itemize}
    \item Unified Model with Symmetry Breaking Constraint
    \item Unified Model with Cumulative Constraint Application
    \item Unified Model with Heule Encoding approach for \texttt{at\_most\_one()}
\end{itemize}
\subsubsection*{Experimental results}

As we can notice from the following table the performances were not very much different from the basic model. Only in the first 10 instances it has been possible to rach at least a suboptimal solution, while for the remaining ones the construction of the model required too much time causing the exceeding of the timeout limit.
\begin{table}[H]
    \centering
    \caption{Objective value through instances}
    \begin{tabular}{cccccc}
            \toprule
            Id & un-model & un-symm-model & un-cum-constr-model & un-heule-enc-model & matrix-model \\ 
            \midrule
            1 & \textbf{14} &       \textbf{14} &   \textbf{14} &   \textbf{14} &   \textbf{14} \\ 
            2 & \textbf{226} &      \textbf{226} &  \textbf{226} &  \textbf{226} &  \textbf{226} \\ 
            3 & \textbf{12} &       \textbf{12} &   \textbf{12} &   \textbf{12} &   \textbf{12} \\ 
            4 & \textbf{220} &      \textbf{220} &  \textbf{220} &  \textbf{220} &  \textbf{220} \\ 
            5 & \textbf{206} &      \textbf{206} &  \textbf{206} &  \textbf{206} &  \textbf{206} \\ 
            6 & \textbf{322} &      \textbf{322} &  \textbf{322} &  \textbf{322} &  \textbf{322} \\ 
            7 & 232 &       238 &   222 &   296 &   292 \\ 
            8 & \textbf{186} &      \textbf{186} &  \textbf{186} &  \textbf{186} &  \textbf{186} \\ 
            9 & \textbf{436} &      \textbf{436} &  \textbf{436} &  \textbf{436} &  \textbf{436} \\ 
            10 & \textbf{244} &     \textbf{244} &  \textbf{244} &  \textbf{244} &  \textbf{244} \\ 
            \bottomrule
    \end{tabular}
\end{table}


\begin{figure}[H]
    \centering
    \begin{subfigure}{0.49\linewidth}
        \centering
        \includegraphics[width=\linewidth]{sat_images/time.pdf}
        \caption{Resolution time for each instance}
    \end{subfigure}
    \hfill
    \centering
    \begin{subfigure}{0.49\linewidth}
        \centering
        \includegraphics[width=\linewidth]{sat_images/conflicts.pdf}
        \caption{Number of conflicts for each instance}
    \end{subfigure}
    % \hfill
    % \centering
    % \begin{subfigure}{0.49\linewidth}
        % \centering
        % \includegraphics[width=\linewidth]{sat_images/max_memory.pdf}
        % \caption{Maximum amount of memory required for each instance}
    % \end{subfigure}
    % \hfill
    % \centering
    % \begin{subfigure}{0.49\linewidth}
        % \centering
        % \includegraphics[width=\linewidth]{sat_images/mk_bool_var.pdf}
        % \caption{Number of boolean variables for each instance}
    % \end{subfigure}
    % \hfill
    % \centering
    % \begin{subfigure}{0.49\linewidth}
        % \centering
        % \includegraphics[width=\linewidth]{sat_images/restart.pdf}
        % \caption{Number of restarts for each instance}
    % \end{subfigure}
    \caption{Statistics about the resolution of the problem for the first 10 instances}
    \label{fig:sat_plots}\footnote{These are the only ones which reach at least a suboptimal solution.}
\end{figure}
    \section{SMT model}

The SMT model is based on the CP model, and follows the same definition of paths of the couriers. 

\subsection{Decision variables}

The SMT model relies on the following decision variables, where boolean variables are from the Propositional Logic Theory, and integer variables are from LIA Theory.

\begin{itemize}
    \item For each package $j$, $A_j \in \{1, \dots, m\}$ (\texttt{ASSIGNMENTS[j]} in Z3) indicates which courier delivers package $j$. More specifically, $A_j = i$ iff the courier $i$ delivers package $j$.
    
    \item For each courier $i$, $P_{i,d} \in \{1, \dots, n + 1\}$ for $d \in \{1, \dots, n + 1\}$ (\texttt{PATH[i][d]} in Z3) models the path taken by the courier $i$. The path is defined such that $P_{i, d_1} = d_1$ iff the location $d_1$ is not part of the route of the courier $i$ and $P_{i, d_1} = d_2$ iff $d_2$ is visited immediately after $d_1$ in the route of $c$. In other words, each relevant location indicates which is its successor and the overall route is defined as a Hamiltonian cycle that starts from $n+1$ (i.e., the depot).

    \item For each courier $i$, $C_i \in \mathbf{N}$ (\texttt{COUNT[i]} in Z3) models the number of packages delivered by courier $i$.
\end{itemize}

\subsection{Objective function}
The objective function is defined as follows:

\[ (\Sum_{j1 \in \{1, \dots, n\} \Sum_{j2 \in \{1, \dots, n\} f(j1, j2)) +  \]


\subsection{Constraints}




\subsection{Validation}
    \section{MIP model}


\subsection{Decision variables}

The MIP models also follow the same idea presented in \Cref{sec:intro}.
% In particular, taking inspiration from the AMPL book \cite{AMPLbook}, we model the Hamiltonian cycle by using a binary tensor that encodes the Hamiltonian cycle of each courier through all the possible delivery points, starting and ending at the depot. 
% This travel is an Hamiltonian cycle and we recall here the definition:
% \begin{definition}[Hamiltonian cycle]
% In the mathematical field of graph theory an Hamiltonian cycle (or Hamiltonian circuit) is a cycle that visits each vertex exactly once.
% \end{definition}
% In order to preserve the properties of the Hamiltonian cycle and to avoid the possible presence of subcircuits we added some constraints, and in particular for the last case we followed the Miller–Tucker–Zemlin (MTZ) approach.\\
The models are based on the following three variables:
\begin{itemize}
    \item A binary tensor $X \in \{0,1\}^{(n+1) \times (n+1) \times m}$, where $X[i,j,k] = 1$ if and only if the courier $k$ departs from the $i$-th delivery point and arrives at the $j$-th one. This formulation is inspired from the AMPL book \cite{AMPLbook}.

    \item A binary matrix $A \in \{0,1\}^{n \times m}$, where $A[i,k] = 1$ if and only if the package $i$ is delivered by the courier $k$. We observe that these variables are not strictly necessary for the model, but they allowed us to write some constraints in an easier way.

    \item An auxiliary matrix $u \in \{1,\dots,n+1\}^{(n+1) \times m}$ that keeps track of the order in which the nodes are visited by each courier starting from $1$. It is necessary, following the MTZ formulation, for subcircuit elimination. The interpretation is that, fixed a courier $k$, $u[i,k] < u[j,k]$ implies that the node $i$ is visited before the node $j$ by the courier $k$.
\end{itemize}


\subsection{Objective function}
As objective function, defined from the assignments as the maximum distance travelled by any courier, we use the following: 
\begin{equation}
    \max_{k \in 1 \dots m}\sum_{i = 1}^{n+1} X[i,j,k] \cdot D[i,j]
\end{equation}
% As upper bound we chose the sum over all the indexes of the matrix $D$: $\sum_{i,j = 1}^{n+1} D[i,j]$. Instead as lower bound we chose the maximum distance travelled picking only one package: $\max_{i \in 1 \dots n} (D[n+1,i] + D[i,n+1])$. In fact, due to the triangular inequality, if another package is picked up by the courier the distance will grow. 

\subsection{Constraints}

We defined the following constraints:
\begin{itemize}
    \item Constraint to link the variables $A$ and $X$:
    \begin{equation}
        \sum_{j = 1}^{n+1} X[i,j,k] = A[i,k]  \qquad  \forall i \in 1 \dots n,\forall k \in 1 \dots m.
    \end{equation}
    \begin{equation}
        \sum_{i = 1}^{n+1} X[i,j,k] = A[j,k]  \qquad  \forall j \in 1 \dots n,\forall k \in 1 \dots m.
    \end{equation}

    \item Constraint to guarantee that each package is assigned:
    \begin{equation}
        \sum_{k = 1}^{m} A[i,k] = 1  \qquad \forall i \in 1 \dots n.
    \end{equation}

    \item Constraint for the capacity of each courier:
    \begin{equation}
        \sum_{i = 1}^{n} A[i,k] s[i] \leq l[k] \qquad \forall k \in 1 \dots m.
    \end{equation}

    \item Constraint to avoid that each courier departs and arrives at the same point:
    \begin{equation}
        X[i,i,k] = 0 \qquad \forall i \in 1 \dots n+1, \forall k \in 1 \dots m.
    \end{equation}

    % \item Two constraints to ensure that there is one arrival and one departure for each node, respectively. This concerns only the internal nodes because the depot is visited by each courier:
    % \begin{equation}
    %     \sum_{i \in 1 \dots n+1, k \in 1 \dots m} X[i,j,k] = 1 \qquad \forall j \in 1 \dots n.  
    % \end{equation}
    % \begin{equation}
    %     \sum_{j \in 1 \dots n+1, k \in 1 \dots m} X[i,j,k] = 1 \qquad \forall i \in 1 \dots n.  
    % \end{equation}

    \item Constraint for the preservation of the flow (if one courier arrives at one node, he departs from the same one):
    \begin{equation}
        \sum_{i = 1}^{n+1} X[i,j,k] = \sum_{i = 1}^{n+1} X[j,i,k] \qquad \forall j \in 1 \dots n, \forall k \in 1 \dots m.
    \end{equation}

    \item Constraints to ensure that each courier starts and ends its route at the depot:
    \begin{equation}
        \sum_{j = 1}^{n} X[n+1,j,k] = 1 \qquad \forall k \in 1 \dots m.
    \end{equation}
    \begin{equation}
        \sum_{j = 1}^{n} X[j,n+1,k] = 1 \qquad \forall k \in 1 \dots m.
    \end{equation}

    \item Constraints for MTZ subtour elimination:
    \begin{itemize}
        \item MTZ condition:
            \begin{equation}
            \makebox[\displaywidth]{$
                u[i,k] - u[j,k] + 1 \leq (n-1)(1 - X[i,j,k]) \qquad \forall i \in 1 \dots n, \forall j \in 1 \dots n+1, \forall k \in 1 \dots m.
            $}
            \end{equation}

        \item $u[i, k] = 1$ iff $i$ is the first item delivered by $k$:
            \begin{equation}
            \makebox[\displaywidth]{$
                u[i,k] \leq X[n+1,i,k] + (n+1)(1-X[n+1,i,k]) \qquad \forall i \in 1 \dots n, \forall k \in 1 \dots m.
            $}
            \end{equation}

        \item $u[j, k] \geq u[i, k] + 1$ iff $i$ is delivered before $j$:
            \begin{equation}
            \makebox[\displaywidth]{$
                u[j,k] \geq (u[i,k] + 1)X[i,j,k] \qquad \forall i \in 1 \dots n, \forall j \in 1 \dots n+1, \forall k \in 1 \dots m.
            $}
            \end{equation}
    \end{itemize}
    
    \subsubsection{Implied Model}
    For the implied model, we added one more constraint with the same meaning of \Cref{eq:impl_constr}:
    \begin{equation}
        \sum_{i = 1}^{n} A[i,k] \geq 1 \qquad \forall k \in 1 \dots m.
    \end{equation}
    
    \subsubsection{Symmetry Model}
    For the symmetry model, we added the symmetry breaking constraint related to the number of delivered packages as defined in \Cref{eq:cp_symm_amount}:
    \begin{equation}
        \sum_{i = 1}^{n} A[i,k] \leq \sum_{i = 1}^n A[i,j],
    \end{equation}
    where $k,j \in 1, \dots, m$ with $k < j$ and $l[k] = l[j]$.
\end{itemize}


\subsection{Validation}

\subsubsection{Experimental design}

For the MIP models, we choose to use the solver-independent language AMPL. The workflow is based on the construction of three different models: the initial one, the implied one, and the symmetry breaking one.

For reproducibility, we decided to only use open-source solvers provided by the AMPL framework such as HiGHS, SCIP, and GCG.


\subsubsection{Experimental results}

Starting from some preliminary experiments, we immediately observed that GCG performs poorly on the first ten instances and therefore decided to discard it from the full experiments.

In \Cref{tab:mip_results}, we present the results of the MIP models. We can observe that the SCIP solver with the addition of implied and symmetry breaking constraints improves in performance. On the other hand, this behavior is not the same for HiGHS.
Nevertheless, the HiGHS solver performances are in general significantly better than the SCIP ones.

\begin{table}[H]
    \centering
    \caption{MIP results. Results in \textbf{bold} are solved to optimality. Instances that are all without a solution have been omitted.}
    \label{tab:mip_results}
    \centerline{
        \begin{tabular}{ccccccc}
            \toprule
            Id & initial-scip & initial-highs & implied-scip & implied-highs & symmetry-scip & symmetry-highs \\ 
            \midrule
            1 & \textbf{14} &       \textbf{14} &   \textbf{14} &   \textbf{14} &   \textbf{14} &   \textbf{14} \\ 
            2 & \textbf{226} &      \textbf{226} &  \textbf{226} &  \textbf{226} &  \textbf{226} &  \textbf{226} \\

            3 & \textbf{12} &       \textbf{12} &   \textbf{12} &   \textbf{12} &   \textbf{12} &   \textbf{12} \\ 
            4 & \textbf{220} &      \textbf{220} &  \textbf{220} &  \textbf{220} &  \textbf{220} &  \textbf{220} \\

            5 & \textbf{206} &      \textbf{206} &  \textbf{206} &  \textbf{206} &  \textbf{206} &  \textbf{206} \\

            6 & \textbf{322} &      \textbf{322} &  \textbf{322} &  \textbf{322} &  \textbf{322} &  \textbf{322} \\

            7 & \textbf{167} &      \textbf{167} &  \textbf{167} &  \textbf{167} &  \textbf{167} &  \textbf{167} \\

            8 & \textbf{186} &      \textbf{186} &  \textbf{186} &  \textbf{186} &  \textbf{186} &  \textbf{186} \\

            9 & \textbf{436} &      \textbf{436} &  \textbf{436} &  \textbf{436} &  \textbf{436} &  \textbf{436} \\

            10 & \textbf{244} &     \textbf{244} &  \textbf{244} &  \textbf{244} &  \textbf{244} &  \textbf{244} \\

            13 & 642 &      726 &   616 &   694 &   526 &   692 \\
            16 & -- &       \textbf{286} &  -- &    557 &   -- &    320 \\
            \bottomrule
        \end{tabular}
    }
\end{table}


For the first ten instances, we plotted the graphs about the execution time (in seconds), the number of simplex iterations, and branching nodes explored by these two solvers.
\begin{figure}[ht]
    \centering
    \begin{subfigure}{0.8\linewidth}
        \centering
        \includegraphics[width=\linewidth]{img/mip/time.pdf}
        % \includegraphics[width=\textwidth]{img/mip/simplex_iterations_2.png}
    \end{subfigure}
    \begin{subfigure}{0.8\linewidth}
        \centering
        % \includegraphics[width=\textwidth]{img/mip/Figure_2.png}
        \includegraphics[width=\linewidth]{img/mip/simplex.pdf}
\end{subfigure}
    \caption{Compared statistics of the performances of the three models}
\end{figure}

% \begin{figure}[H]
%     \centering
%     \includegraphics[width=\linewidth]{img/mip/simplex.pdf}
%     \caption{Compared statistics of the performances of the three models, divided by SCIP and HiGHS solvers.}
% \end{figure}


On a theoretical level, SCIP and HiGHS try to initially solve the relaxed-version of the problem (LP) using the revised simplex-method and finding a lower bound for the solution; then, if the solution found is not an integer, they start to solve the MIP part using branch-and-cut (SCIP) or branch-and-bound (HiGHS). For the resolution of the sub-problems generated by these two methods, they proceed recursively by applying the same algorithm. Therefore, from the plots, we can observe that a lower number of simplex iterations and branching nodes corresponds to a faster resolution time. This is in line with our previous observation regarding HiGHS as a better performing solver.



    \section{Conclusions}

    We experimented several models using different methods and attempted to implement the same core idea across the whole project to make results comparable. Overall, all approaches are able to solve the smaller instances while, for bigger instances, only CP and SMT were able to at least produce a suboptimal solution by using proper search heuristics. Moreover, for this problem, we surprisingly noted that symmetry breaking constraints generally tend to, except in a few cases, worsen the results. To conclude, we can observe that, for this formulation of the problem, being able to guide the solver when exploring the search space is one of the most important factors to obtain good results.

    \printbibliography

    \begin{appendices}
        \section{Implied constraint proof} \label{sec:impl_proof}
        \begin{lem}
            Assuming that the capacity of each courier allows delivering at least a package, if there exists an optimal solution, then there exists an optimal solution where each courier delivers at least one package.
        \end{lem}
        \begin{proof}
            Let us assume that the optimal solution is $D_j$ and there is a courier, say $k_1$, which do not deliver any package. Let us also suppose that the courier $k_j$ is the one that covers the maximum distance $D_j$. If we assign one package that $k_j$ brings, say $i$, to $k_1$, then, due to the triangle inequality, the two new distances $D_1$, travelled by the courier $k_1$ with $i$, and $D_2$, travelled by the courier $k_j$ without $i$, are less or equal to $D_j$. In fact:
            \begin{equation}
                \begin{split}
                    D_1 = &D[\texttt{depot},i] + D[i,\texttt{depot}] \leq\\
                        & D[\texttt{depot}, i_1] + \dots + D[i_r, i] + D[i, i_s] + \dots + D[i_t, \texttt{depot}] = D_j.
                \end{split}
            \end{equation}
            \begin{equation}
                \begin{split}
                    D_2 = &D[\texttt{depot},i_1] + \dots + D[i_r,i_s] + \dots + D[i_t, \texttt{depot}] \leq\\
                    &D[\texttt{depot}, i_1] + \dots + D[i_r, i] + D[i, i_s] + \dots + D[i_t, \texttt{depot}] = D_j.
                \end{split}
            \end{equation}
            Therefore, there are the following cases:
            \begin{itemize}
                \item If $D_1 = D_2$, either both $k_1$ and $k_j$ cover the maximum distance or neither of them do, and another courier has a route of cost $D_j$.
                \item If $D_1 > D_2$, either $k_1$ covers the maximum distance or another courier that is not $k_1$ and $k_j$ does.
                \item If $D_1 < D_2$, same as above for $k_j$.
            \end{itemize}
            So, an optimal solution still exists and has as objective value $D_j$.
        \end{proof}
    \end{appendices}

\end{document}